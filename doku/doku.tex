\documentclass[a4paper,12pt]{article}

% Pakete laden
\usepackage[utf8]{inputenc} % Zeichencodierung
\usepackage[T1]{fontenc}    % Ausgabe von Umlauten
\usepackage[ngerman]{babel} % Deutsche Silbentrennung und Sprache
\usepackage{amsmath}        % Mathematische Symbole
\usepackage{graphicx}       % Einfügen von Bildern
\usepackage{hyperref}       % Für Hyperlinks

% Dokumentbeginn
\begin{document}

% Titel, Autor, Datum
\title{Projektarbeit: Praxissetellenbewerten}
\author{David Damke}
\date{\today}

\maketitle

% Inhaltsverzeichnis (optional)
\tableofcontents
\newpage

% Abschnitt 1
\section{Einführung}
In dieser Pdf wird gezeigt wie ein Vue App mit Node Backend auf einem Debian Server eingerichtet werden müssen. 
Dazu noch welche Schritte für das Rechenzentrum Notwendig sind und welche Versionen und Plugins verwendet wurden. 


% Abschnitt 2
\section{Plugins und Versionen}

\subsection{NodeJs}
NodeJs : v20.11.0
"cors": "2.8.5",
"express": "4.18.2",
"ldapjs": "3.0.7",
"mongodb": "6.3.0"


\subsection{Vue}
axios 1.6.7,
vue": "3.3.11",
vue-router": "4.2.5",
vuetify": "3.0.1",
vuex": "4.1.0"

"@mdi/font": "7.4.47",
    "@vitejs/plugin-vue": "4.5.2",
    "vite": "5.0.10"

\section{Rechenzentrum}

Rechenzentrum muss die eignenen Firewall Regeln so anpassen das die ServerIp auf den Ldap Server zugreiffen kann.




% Abschnitt 3
\section{Einrichtung auf Debian Webserver}

\subsubsection{Verbindung zum Server}

ssh name@ip
password : 

\subsection{Apache2}

\subsubsection{Webapp}

npm run build 

dist ordner mittels scp an Server schicken 

unter /var/www/ Speichern 

Unter /etc/apach2/ .... neue Conf anlegen

Hier beschreiben was in Conf rein muss. 





\subsubsection{Node Backend}

Auf Server 

Als Service von Ubuntu Starten 

unter /etc/apache2  ein Reverse Proxy erstellen 



\end{document}
